\section{Sequences and Series of Functions}
We will consider how to extend the idea of a sequence of numbers to a sequence of functions $(f_1,f_2,f_3,\ldots)$.
\par\vspace{0.6 cm}
To simplify things, we'll let the domain of each function $f_n$ is the same set $A$.
\par\vspace{0.6 cm}
\begin{definition*}[pointwise convergence] For each $n\in\mathbb{N}$ let $f_n$ be a function defined on a set $A\subseteq\mathbb{R}$. The sequence $(f_n)$ of functions \textit{converges pointwise on $A$} to a function $f$ if, for all $x\in A$, the sequence of real numbers $f_n(x)$ converges to $f(x)$. 
\end{definition*}
\par\vspace{0.6 cm}
\begin{example*}
Let 
\[
f_n(x) = (x^2+nx)/n\quad x\in\mathbb{R}
\]
Then
\[
f_1(x) = x^2+x,\quad f_2(x) = \frac{x^2+2x}{2},\quad f_3(x)=\frac{x^2+3x}{3},\ldots
\]
and
\[
f_n(x) = \frac{x^2+nx}{n} = \frac{x^2}{n}+x \rightarrow x\quad\mbox{as}\quad n\rightarrow\infty
\]
\par\vspace{0.6 cm}
Therefore, $(f_n)$ converges pointwise to $f(x)=x$.
\end{example*}
\begin{example*}
Let $g_n(x)=x^n$ for $x\in[0,1]$.  For $x\in[0,1)$, 
\[
g_n(x)\rightarrow 0\quad\mbox{as}\quad n\rightarrow0
\]  
but if $x=1$, $g_n(1)=1$ for all $n\in\mathbb{N}$, so
\[
\lim_{n\rightarrow\infty}g_n(x) = g(x) =\left\{\begin{array}{lcl}0 &\mbox{if}& 0\leq x <1\\1 &\mbox{if}& x=1\end{array}\right.
\]
\par\vspace{0.6 cm}
Note that each $g_n$ is continuous on $[0,1]$, but $g$ is not.  This illustrates the fact that the pointwise limit of a sequence of continuous functions may not be continuous.
\end{example*}
\par\vspace{0.6 cm}
\begin{example*}
Let 
\[
h_n(x) = x^{1+\frac{1}{2n-1}}\quad x\in[-1,1]
\]
Then
\[
h_1(x) = x\cdot x,\quad h_2(x) = x\cdot x^{\frac{1}{3}},\quad h_2(x) = x\cdot x^{\frac{1}{5}}, \quad
\]
For a fixed $x$, 
\[
\lim_{n\rightarrow\infty} = x\lim_{n\rightarrow\infty}x^{\frac{1}{2n-1}} = |x|
\]
\par\vspace{0.6 cm}
In this example, for each $n$, $g_n(x)$ is differentiable on $[-1,1]$, but the pointwise limit $g(x)$ is not.
\end{example*}
\par\vspace{0.3 cm}
\begin{theorem*}[Cauchy criterion for uniform convergence]  A sequence of functions $(f_n)$ defined on $A\subseteq\mathbb{R}$ converges uniformly on $A$ if and only if, for every $\epsilon>0$, there is an $N\in\mathbb{N}$ such that
\[
|f_n(x)-f_m(x)|<\epsilon\quad\mbox{whenever}\quad n,m\geq N\quad\mbox{and}\quad x\in A
\]
\end{theorem*}
\begin{proof}
$(\Leftarrow)$ Suppose $(f_n)$ is a series of functions on $A\subseteq\mathbb{R}$ that converges uniformly on $A$ to $f$.  Let $\epsilon>0$ be given.  Then by the definition of uniform convergence there exists an $N\in\mathbb{N}$ such that
\[
|f_k(x)-f(x)|<\frac{\epsilon}{2}\quad\mbox{whenever}\quad k\geq N\quad\mbox{and}\quad x\in A 
\]
and, for $n,m\geq N$, 
\[
|f_n(x)-f_m(x)| = |f_n(x)-f(x)+f(x)-f_m(x)| \leq |f_n(x)-f(x)|+|f_m(x)-f(x)| < \frac{\epsilon}{2}+\frac{\epsilon}{2} = \epsilon
\]
whenever $n,m\geq N$ and $x\in A$.
\par\vspace{0.6 cm}
$(\Rightarrow)$ Now suppose $(f_n)$ is a sequence of functions defined on $A\subseteq\mathbb{R}$ with the property that, for any $\epsilon>0$, there exists an $n\in\mathbb{N}$ such that
\[
|f_n(x)-f_m(x)|<\epsilon\quad\mbox{whenever}\quad m,n\geq N\quad\mbox{and}\quad x\in A
\]  
Let $\epsilon>0$ be given.  By hypothesis there is an $N\in\mathbb{N}$ such that
\[
|f_n(x)-f_m(x)| < \frac{\epsilon}{2}\quad\mbox{whenever}\quad n\geq N\quad\mbox{and}\quad x\in A
\]
For any $x\in A$, we can say that $(f_n(x))$ is a Cauchy sequence in $\mathbb{R}$, and by the Cauchy criterion for real sequences, for any fixed $x\in A$, the sequence $(f_n(x))$ converges.  For every $x\in A$, define $f(x)$ to be the 
limit of the sequence $f_n(x)$.  Then because $f(x)$ is the limit of the sequence $(f_n(x))$, there is 
an $N_x\in\mathbb{N}$ such that
\[
|f_k(x)-f(x)|<\epsilon\quad\mbox{whenever}\quad k\geq N_x\quad\mbox{with}\quad N_x\geq N
\]   
(note: we can assume without loss of generality that $N_x\geq N$ because even if a smaller value will work, so will any larger value, so we can always increase the original $N_x$ to $N$)
\par\vspace{0.5 cm} 
Then for $n\geq N$, for any $x\in A$,
\[
|f_n(x)-f(x)| = |f_n(x)-f_{N_x}(x)+f_{N_x}(x)-f(x)|
\]
\[
\leq |f_n(x)-f_{N_x}(x)| + |f_{N_x}-f(x)| < \frac{\epsilon}{2}+\frac{\epsilon}{2} = \epsilon
\]
Since this is true for any $x\in A$, $(f_n)$ converges uniformly to $f$. (Note: the $N_x$ values may depend on $x$, but they are always greater than or equal to $N$, and for this reason no matter what $N_x$ is,
\[
|f(x_n)-f_{N_x}|<\frac{\epsilon}{2}
\]
for any $N_x>N$.  The particular choice of $N_x$ does not matter, and we can always choose $N_x$ large enough to make the other term smaller than $\epsilon/2$.

\end{proof}
\begin{conjecture*}
If $f_n\rightarrow f$ pointwise on a compact set $K$, then $f_n\rightarrow f$ uniformly on $K$.
\end{conjecture*}
\begin{remark}
The conjecture is false.  Consider the sequence from an earlier example:
\[
f_n(x) = x^n\quad x\in[0,1]
\]
In this case $K=[0,1]$ is compact, and $f_n\rightarrow f$ pointwise on $K$ with
\[
f(x)=\left\{\begin{array}{lcl}
0 &\mbox{if}& x<1\\
1 &\mbox{if}& x=1\end{array}\right\}
\]
In this case the limit of a sequence of continuous functions on a compact set is not continuous.  We have a theorem stating that the uniform limit of a sequence of continuous functions is continuous, so the convergence cannot be uniform.
\end{remark}
\begin{conjecture*}
If $f_n\rightarrow f$ on $A$ and $g$ is bounded on $A$, then $f_ng\rightarrow fg$ on $A$.
\end{conjecture*}
\begin{proof}
Let $\epsilon>0$ be given.  By hypothesis, $g$ is bounded on $A$, so there is an $M\in\mathbb(0,\infty)$ such that
\[
|g(x)|\leq M\quad\mbox{for all}\quad x\in A
\]
We have to show that there exists an $N\in\mathbb{N}$ such that when $n\geq N$,
\[
|f_ng-fg|<\epsilon
\]
By hypothesis, $f_n\rightarrow f$ uniformly on $A$, so there is an $N\in\mathbb{N}$ such that
\[
|f_n-f| < \frac{\epsilon}{M}\quad\mbox{whenever}\quad n\geq N
\]
Then for $n\geq N$,
\[
|f_ng-fg| =|g||f_n-f| \leq M|f_n-f| < M\frac{\epsilon}{M} = \epsilon
\]
\end{proof}
\begin{conjecture*}
If $f_n\rightarrow f$ uniformly on $A$ and each $f_n$ is bounded, then $f$ is bounded.
\end{conjecture*}
\begin{proof}
By hypothesis, $f_n\rightarrow f$ uniformly on $A$.  By definition, for $\epsilon=1$, there exists an $N\in\mathbb{N}$ such that
\[
|f_N-f| < 1
\]
(this is a special case of the statement than $|f_n-f|<\epsilon$ when $n\geq N$)
By hypothesis, $f_N$ is bounded, so there is an $M\in(0,\infty)$ such that
\[
|f_N(x)| < M\quad\mbox{for all}\quad x\in A
\]
Then for all $x\in A$,
\[
|f(x)| = |-f(x)| = |f_N(x)-f(x)+f_N(x)| \leq |f_N(x)-f(x)|+|f_N(x)| < 1+M 
\]
so $f$ is bounded.
\end{proof}
\begin{conjecture*}
If $f_n\rightarrow f$ uniformly on $A$ and $f_n\rightarrow f$ uniformly on $B$, then $f_n\rightarrow f$ uniformly on $A\cup B$.
\end{conjecture*}
\begin{proof}
Let $\epsilon>0$ be given.  By hypothesis, $f_n\rightarrow f$ uniformly on $A$, so there is an $N_A\in\mathbb{N}$ such that for all $n\in\mathbb{N}$ and all $x\in A$,
\[
|f_n-f|<\epsilon\quad\mbox{whenever}\quad n\geq N_A
\]
Also by hypothesis, $f_n\rightarrow f$ uniformly on $B$, so there is an $N_B\in\mathbb{N}$ such that for all $n\in\mathbb{N}$ and all $x\in B$,
\[
|f_n-f|<\epsilon\quad\mbox{whenever}\quad n\geq N_B
\]
Let $N$ be the larger of $N_A$ and $N_B$.  Then for all $x\in A\cup B$,
\[
|f_n(x)-f(x)| < \epsilon\quad\mbox{whenever}\quad n\geq N
\]
\end{proof}
\begin{conjecture*}
If $f_n\rightarrow f$ uniformly on an interval $A$ and if each $f_n$ is increasing, then $f$ is increasing.
\end{conjecture*}
\begin{proof}
Let $a,b$ be arbitrary points in $A$ with $a<b$.  By hypothesis, $f_n$ is increasing for every $n\in\mathbb{N}$, so we can write
\[
f_n(a) \leq f_n(b)\quad\mbox{for all}\quad n
\]
By the order limit theorem, this means
\[
\lim f_n(a) \leq \lim f_n(b)
\]
But $\lim f_n=f$, so $f_n(a)\rightarrow f(a)$ and $f_n(b)\rightarrow f(b)$, and therefore
\[
f(a) \leq f(b)
\]
\end{proof}
\begin{conjecture*}
If $f_n\rightarrow f$ pointwise on an interval $A$ and if each $f_n$ is increasing, then $f$ is increasing.
\end{conjecture*}
\begin{proof}
The proof in the case of uniform convergence did not use uniform convergence, only pointwise convergence, so this is true by the same proof.
\end{proof}
\begin{conjecture*}
Let $f_0(x)=x$ for $x\in[0,1]$.
\par\vspace{0.3 cm}
Now let
\[
f_1(x) =\left\{\begin{array}{lcl}
(3/2)x &\mbox{if}& 0\leq x\leq 1/3\\
1/2 &\mbox{if}& 1/3 < x < 1/3\\
(3/2)x-1/2 &\mbox{if}& 2/3\leq x\leq 1\end{array}\right.
\]
Then $f$ is continuous and increasing on $[0,1]$, and constant on the middle third.
\end{conjecture*}
\begin{conjecture*}
Let
\[
f_2(x) =\left\{\begin{array}{lcl}
f_1(3x)/2 &\mbox{if}& 0\leq x\leq 1/3\\
f_1(x) &\mbox{if}& 1/3 < x < 1/3\\
f_1(3x-2)/2-1/2 &\mbox{if}& 2/3\leq x\leq 1\end{array}\right.
\]
If we continue this process, $f_n\rightarrow f$ uniformly on $[0,1]$.
\end{conjecture*}

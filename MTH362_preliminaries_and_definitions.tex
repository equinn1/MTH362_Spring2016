\begin{definition*}[binary operation] A \textbf{binary operation} on a set $S$ is a function from $S\times S$ into $S$.
\par\vspace{0.3 cm}
Examples of binary operations:
\begin{itemize} 
\item $+:\mathbb{N}\times\mathbb{N} \rightarrow \mathbb{N}$ Addition of natural numbers
\item $\cdot : \mathbb{N}\times\mathbb{N} \rightarrow \mathbb{N}$ Multiplication of natural numbers
\end{itemize}
\par
\end{definition*}
\par\vspace{1 cm}
\begin{definition*}[group] A \textbf{group} consists of:
\begin{itemize} 
\item A set $G$
\item A binary operation $+ : G\times G\rightarrow G$ with the following properties:
\par
\begin{tabular}{ll}
$x+(y+z)=(x+y)+z\;\forall x,y,z\in G$ & (associativity)\\
$\exists 0\in G\;\mbox{ such that } a+0=0+a=a\;\forall a\in G$ & (identity)\\
$\forall a\in G\;\exists\;a^{-1}\;\mbox{ such that }\; a+a^{-1}=a^{-1}+a=0$ & (inverse)\\ 
\end{tabular}
\end{itemize}
\end{definition*}
\par\vspace{1 cm}
\begin{definition*}[field] A \textbf{field} consists of:
\begin{itemize} 
\item A set $F$
\item A binary operation $+ : F\times F\rightarrow F$ with the following properties:
\par
\begin{tabular}{ll}
$x+y=y+x\;\forall x,y\in F$ & (additive commutativity)\\
$x+(y+z)=(x+y)+z\;\forall x,y,z\in F$ & (additive associativity)\\
$\exists 0\in F\;\mbox{ such that } a+0=0+a=a\;\forall a\in F$ & (additive identity)\\
$\forall a\in F\;\exists\;a^{-1}\;\mbox{ such that }\; a+a^{-1}=a^{-1}+a=0$ & (additive inverse)\\ 
\end{tabular}
\item A binary operation $: F\times F\rightarrow F$ with the following properties:
\par
\begin{tabular}{ll}
$xy=yx\;\forall x,y\in F$ & (multiplicative commutativity)\\
$x(yz)=(xy)z\;\forall x,y,z\in F$ & (multiplicative associativity)\\
$\exists 1\in F\;\mbox{ such that } a1=1a=a\;\forall a\in F$ & (multiplicative identity)\\
$\forall a\in F\setminus 0\;\exists\;a^{-1}\;\mbox{ such that }\; aa^{-1}=a^{-1}a=1$ & (multiplicative inverse)\\ 
$x(y+z)=xy+xz\quad\forall x,y,z\in F$ & (distributive property)\\
\end{tabular}
\end{itemize}
\end{definition*}
\par\vspace{1 cm}
\begin{definition*}[vector space] A \textbf{vector space} or \textbf{linear space} consists of:
\begin{itemize}
\item A field $F$ of elements called \textbf{scalars}
\item A commutative group $V$ of elements called \textbf{vectors} with respect to a binary operation $+$
\item A binary operation $:F\times V\rightarrow V$ called \textbf{scalar multiplication} that associates 
with each scalar $\alpha\in F$ and vector $v\in V$ a vector $\alpha v$ in such a way that:
\par
\begin{tabular}{l}
$1v=v\quad \forall v\in V$\\
$(\alpha\beta)v=\alpha(\beta v)\quad\forall \alpha,\beta\in F,\;v\in V$\\
$\alpha(v+w)=\alpha v+\alpha w\quad\forall\alpha\in F,\; v,w\in V$\\
$(\alpha+\beta)v=\alpha v+\beta v\quad\forall \alpha,\beta\in F,\;v\in V$\\
\end{tabular}
\end{itemize}
\end{definition*}
\begin{definition*}[norm] A nonnegative real-valued function $\|\;\|:V\rightarrow\mathbb{R}$ is called a \textbf{norm} if:
\begin{itemize}
\item $\|v\|\geq 0\mbox{ and }\|v\|=0\;\Leftrightarrow\;v=\vec{0}$
\item $\|v+w\|\leq\|v\|+\|w\|\quad$ (triangle inequality)
\item $\|\alpha v\|=|\alpha|\,\|x\|\quad\forall\alpha\in F,\;v\in V$
\end{itemize}
\end{definition*}
\par\vspace{1 cm}
\begin{definition*}[normed linear space] 
A linear space $V$ together with a norm $\|\cdot\|$, denoted by the pair $(V,\|\cdot\|$), is called a \textbf{normed linear space}
\end{definition*}

\begin{definition*}[inner product] Let the field $F$ be either $\mathbb{R}$ or $\mathbb{C}$ and a set $V$ of vectors which together with $F$ form a vector space.  An \textbf{inner product} on $V$ is a map 
\[
\cdot : V\times V\rightarrow\mathbb{F}
\vspace{0.3 cm}
\]
with the following properties:
\par\vspace{0.2 cm}
\begin{tabular}{ll}
$(u+v)\cdot w = u\cdot w\;+\;v\cdot w $ & $\forall u,v,w\in V$\\
$(\alpha u)\cdot v=\alpha(u\cdot v)$ & $\forall \alpha\in F,\;u,v\in V$\\
$u\cdot v = (\overline{v\cdot u})$  & $\forall u,v\in V$\\
$u\cdot u\geq 0$ & $\forall u\in V$ with equality when $u=\vec{0}$\\ 
\end{tabular}
\par\vspace{0.3 cm}
If the underlying field is $\mathbb{R}$, the fourth condition can be replaced by
\par\vspace{0.2 cm}
\begin{tabular}{ll}
$u\cdot v = v\cdot u$  & $\forall u,v\in V$\\
\end{tabular}
\par\vspace{0.2 cm}
since a real number is its own conjugate.  In this case, the condition just says the inner product is commutative.
\end{definition*}
\par\vspace{1 cm}
\begin{definition*}[metric] A \textbf{metric} on a set $S$ is a function
\[
\rho:S\times S\rightarrow\mathbb{R}
\vspace{0.3 cm}
\]
where $\rho$ has the following three properties for any $x,y,z\in S$: 
\[
\begin{array}{l}
\rho(x,y)\geq0\;\mbox{ and } \rho(x,y)=0\Leftrightarrow x=y\\
\rho(x,y)=\rho(y,x)\\
\rho(x,y)\leq \rho(x,z)+\rho(z,y)\\
\end{array}
\vspace{0.3 cm}
\]
\end{definition*}
\par\vspace{1 cm}
\begin{definition*}[metric space] A \textbf{metric space} is a pair $\{S,\rho\}$ where $S$ is a set and $\rho$ is a metric defined on $S$.
\end{definition*}
\par\vspace{1 cm}
\begin{definition*}[topology] A \textbf{topology} is a set $X$ and a collection $\mathcal{J}$ of subsets of $X$ having the following properties:
\begin{itemize}
\item $\emptyset$ and $X$ are in $\mathcal{J}$
\item The union of any subcollection of elements of $\mathcal{J}$ belongs to $\mathbb{J}$
\item The intersection of any \textit{finite} subcollection of $\mathcal{J}$ belongs to $\mathcal{J}$
\end{itemize} 
\end{definition*}


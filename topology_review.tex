\section*{Convergence}
\begin{definition*}[sequence] A \textit{sequence} is a function whose domain is $\mathbb{N}$.
\end{definition*}
\par\vspace{1 cm}
\begin{definition*}[convergent sequence] A sequence ${x_n}$ in $\mathbb{R}$ \textit{converges} to $x$ if, for every $\epsilon>0$, there is an $N\in\mathbb{N}$ such that
\[
|x_n-x| , \epsilon\quad\mbox{whenever}\quad n\geq N
\]
\end{definition*}
\begin{definition*}[Cauchy sequence] A sequence ${x_n}$ in $\mathbb{R}$ is said to be a \textit{Cauchy} sequence if, for every $\epsilon>0$, there is an $N\in\mathbb{N}$ such that
\[
|x_n-x_m| , \epsilon\quad\mbox{whenever}\quad n,m\geq N
\]
\end{definition*}
\par\vspace{1 cm}
\section*{Topology Review}
\begin{definition*}[$\epsilon$-neighborhood] If $x\in\mathbb{R}$ and $\epsilon>0$, the $\epsilon$-neighborhood $V_{\epsilon}(x)$ is defined by 
\[
V_{\epsilon} = \left\{y\in\mathbb{R}\;:\; |x-y|<\epsilon\right\}
\]
\end{definition*}
\par\vspace{1 cm}
\begin{definition*}[limit point] A real number $x\in A$ is a \textit{limit point} of $A$ if for every $\epsilon>0$, $V_{\epsilon}(x)$ contains elements of $A$ other than $x$.
\end{definition*}
\par\vspace{1 cm}
\begin{definition*}[isolated point] A element $x\in A$ is an \textit{isolated point} of $A$ if $x\in A$ and $x$ is not a limit point of $A$.
\end{definition*}
\par\vspace{1 cm}
\begin{definition*}[open set] A set $A\subseteq\mathbb{R}$ is \textit{open} if, for every $a\in A$, there is an $\epsilon>0$ such that
\[
V_{\epsilon}(a)\subseteq A
\]
That is, there is an $\epsilon$-neighborhood of every element of $A$ that is contained entirely in $A$.
\end{definition*}
\par\vspace{1 cm}
\begin{definition*}[closed set] A set $A\subseteq\mathbb{R}$ is \textit{closed} if it contains all of its limit points.
\end{definition*}
\par\vspace{1 cm}
\begin{definition*}[closure] The \textit{closure} of a set $A\subseteq\mathbb{R}$ denoted by $\overline{A}$ is the union of $A$ and its limit points.
\end{definition*}
\par\vspace{1 cm}
\begin{definition*}[compact set] A set $A\subseteq\mathbb{R}$ is \textit{compact} if every sequence in $A$ has a convergent subsequence whose limit is in $A$.
\end{definition*}
\par\vspace{1 cm}
\begin{definition*}[perfect set] A set $A\subseteq\mathbb{R}$ is \textit{perfect} if it is closed and has no isolated points.
\end{definition*}
\par\vspace{1 cm}
\begin{definition*}[bounded set] A set $A\subseteq\mathbb{R}$ is \textit{bounded} if there exists an $M>0$ such that
\[
|a| \leq M\quad\mbox{for all}\quad a\in A
\]
\end{definition*}
\par\vspace{1 cm}
\begin{definition*}[separated sets] Two nonempty sets $A,B\subseteq\mathbb{R}$ are \textit{separated} if
\[
A\cap\overline{B} = \emptyset = \overline{A}\cap B
\]
\end{definition*}
\par\vspace{1 cm}
\begin{definition*}[disconnected set] A set $E\subseteq\mathbb{R}$ is \textit{disconnected} if it can be written as 
\[
E=A\cup B
\]
where $A$ and $B$ are nonempty separated sets.
\end{definition*}
\par\vspace{1 cm}
\begin{definition*}[connected set] A set $A\subseteq\mathbb{R}$ is \textit{connected} if it is not disconnected.
\end{definition*}
\par\vspace{1 cm}
\begin{theorem*} A set $A\subseteq\mathbb{R}$ is open if and only if its compliment $A^c$ is closed.
\end{theorem*}
\begin{proof}
$(\Rightarrow)$ Let $A\subseteq\mathbb{R}$ be open.  Suppose for the sake of contradiction that $a\in A$ is a limit point of $A^c$. By definition, for every $\epsilon>0$, $V_{\epsilon}(a)$ contains points of $A^c$.  This means that no $\epsilon$-neighborhood of $a$ is entirely contained in $A$, contradicting the assumption that $A$ is open.
\par\vspace{0.3 cm}\noindent
$(\Leftarrow)$ Now suppose $A\subseteq\mathbb{R}$ with $A^c$ closed.  Let $a$ be an element of $(A^c)^c=A$.  By hypothesis, $A^c$ is closed, so $a$ cannot be a limit point of $A^c$ because $a\notin A$.  By definition, this means there must be an $\epsilon>0$ such that $V_{\epsilon}(a)$ contains no points of $A^c$.  Therefore, $V_{\epsilon}(a)\subseteq A$.  Since $a$ was arbitrarily chosen, we can find an $\epsilon$-neighborhood of every element of $A$ that is entirely contained in $A$, so by definition $A$ is open.  
\end{proof}
\par\vspace{1 cm}
\begin{theorem*}[Heine-Borel Theorem] A set $K\subseteq\mathbb{R}$ is compact if and only if it is closed and bounded.
\end{theorem*}
\begin{proof}
$(\Rightarrow)$ Let $K\subseteq\mathbb{R}$ be compact.  Suppose for the sake of contradiction that $K$ is not bounded. By definition, for any $M\in\mathbb{R}$, there exists an element $x_M$ of $K$ with $|x_M|>M$.  We will use this fact to construct a sequence in $K$ that has no convergent subsequence.  If $K$ is unbounded, there must be an $x_1\in K$ with $|x_1|>1$.  By the same argument, there must be an element $x_2$ in $K$ with $|x_2|>2$.  Continuing in this fashion, we may construct a sequence
\[
x_1,x_2,x_3,\ldots,x_n,\ldots\quad\mbox{with}\quad |x_n|>n\quad\mbox{for every}\quad n\in\mathbb{N}
\]   
Now suppose $(x_{n_k})$ is a subsequence of $(x_n)$.  Since $|x_{n_k}|>n_k$ and $n_k$ is unbounded, every subsequence is unbounded.  Since all convergent sequences are bounded, this means that no subsequence is convergent, contradicting the assumption that $K$ is compact.
\par\vspace{0.3 cm}
This establishes that $K$ is bounded.  To show that $K$ is closed, suppose $x$ is a limit point of $K$.  Then there is a sequence $(x_n)$ in $K$ that converges to $x$.  By hypothesis, $K$ is compact, so by definition $(x_n)$ has a convergent subsequence whose limit is in $K$.  However, by an earlier theorem, any subsequence of a convergent sequence is also convergent, and has the same limit $x$.  Therefore both $(x_n)$ and its subsequence have the same limit, which must belong to $K$.  Since $x$ was arbitrarily chosen, this is true of any limit point, so $K$ must contain all of its limit points and therefore is closed.
\par\vspace{0.3 cm}
$(\Leftarrow)$ Now suppose $K$ is closed and bounded.  Let $(x_n)$ be an arbitrary sequence in $K$.  Then $(x_n)$ must be bounded, and by the Bolzano-Weierstrass theorem, it must have a convergent subsequence $(x_{n_k})$.  The limit of this subsequence is, by definition, a limit point of $K$, and by hypothesis $K$ is closed and therefore contains its limit points.  Therefore, every sequence in $K$ has a subsequence that converges to a point in $K$, and by definition, $K$ is compact.
      
\end{proof}
\par\vspace{1 cm}
\begin{theorem*} A set $E\subseteq\mathbb{R}$ is connected if and only if, for all nonempty disjoint sets $A$ and $B$ satisfying $E=A\cup B$ there always exists a convergent sequence $(x_n)\rightarrow x$ with $(x_n)$ contained in one of $A$ and $B$, and $x$ an element of the other.
\end{theorem*}
\begin{proof}
$(\Rightarrow)$ Let $E$ be a connected set. Suppose $E=A\cup B$ where $A$ and $B$ are disjoint, nonempty sets.  By hypothesis, $E$ is connected, so $A$ and $B$ are not separated. This means that one of $A\cap\overline{B}$ and $\overline{A}\cap B$ is not empty.  Without loss of generality, assume that $x\in\overline{A}\cap B$.  Then $x\in\overline{A}$ and $x\in B$.  But $A$ and $B$ are disjoint, so $x\notin A$.  By definition, $\overline{A}$ is the union of $A$ and its limit points, and since $x\notin A$, $x$ must be a limit point of $A$.  By an earlier theorem, there is a sequence in $A$ that converges to $x$.    
\par\vspace{1 cm}
$(\Leftarrow)$ (contrapositive argument) Suppose $E\subseteq\mathbb{R}$ is disconnected.  We need to find two nonempty, disjoint sets $A$ and $B$ such that $E=A\cap B$ and there does not exist a convergent sequence $(x_n)$ in $A$ with its limit $x$ in $B$, or vice-versa.  By hypothesis, $E$ is separated, so there exist separated sets $A$ and $B$ with $E=A\cup B$. Now suppose $(x_n)$ is a convergent sequence contained in $A$ whose limit is $x$.  By definition, since $x$ is a limit of a sequence in $A$, $x$ belongs to $\overline{A}$. Because $A$ and $B$ are disconnected, by definition $\overline{A}\cap B$ is empty, so $x\notin B$.  Since $(x_n)$ was an arbitrary sequence, no convergent sequence in $A$ has its limit in $B$.  A similar argument shows that no convergent sequence in $B$ has its limit in $A$. 
\end{proof}
\par\vspace{1 cm}
\begin{definition*}[continuous function] In a metric space $(X,\rho)$ a function $f:D\subseteq X\rightarrow X$ is continuous if, for every $\epsilon>0$, there is a $\delta>0$ such that
\[
\rho(f(x),f(y))<\epsilon\quad\mbox{whenever}\quad\rho(x,y)<\delta
\]
\end{definition*}
\par\vspace{1 cm}
\begin{lemma*}
If $(X,\rho)$ is a metric space and $f:D\subseteq X\rightarrow X$ is continuous, and $(x_n)$ is a convergent sequence in $D$ whose limit $x$ is also in $D$, then $\lim f(x_n)=f(x)$.
\end{lemma*}
\begin{proof}
Let $\epsilon>0$ be given.  We need to show that there is an $N\in\mathbb{N}$ such that
\[
\rho(f(x_n),f(x))<\epsilon\quad\mbox{whenever}\quad n\geq N
\]
By hypothesis, $f$ is continuous, so by definition for every $\epsilon>0$, there is a $\delta>0$ such that
\[
\rho(f(x_n),f(x))<\epsilon\quad\mbox{whenever}\quad\rho(x,x_n)<\delta
\]
Also by hypothesis, $(x_n)$ is convergent with limit $x$, so by definition there is an $N\in\mathbb{N}$ such that 
\[
\rho(x_n,x)<\delta\quad\mbox{whenever}\quad n\geq N
\]
But by the continuity of $f$, this is equivalent to saying that $\rho(f(x_n),f(x))<\epsilon$ whenever $n\geq\mathbb{N}$.
\end{proof}
\par\vspace{1 cm}
\begin{theorem*} If $f:\mathbb{R}\rightarrow\mathbb{R}$ and $K\subseteq\mathbb{R}$ is compact, then $f[K]$, the image of $K$ under $f$, is also compact.
\end{theorem*}
\begin{proof}
Let $y_n$ be a sequence in $f[K]$.  By hypothesis, $y_n\in f[K]$ for each $n\in\mathbb{N}$ so there is at least one $x\in K$, call it $x_n$, with $f(x_n)=y_n$.  Because $K$ is compact, the sequence $(x_n)$ in $K$ must have a convergent subsequence $(x_{n_k})$ whose limit $x$ is in $K$. By the previous lemma, with the hypothesis that $f$ is continuous it must be true that that $(y_{n_k})$ is convergent with limit $y=f(x)$.  Since $(y_n)$ was an arbitrary sequence in $f(K)$, it follows that every sequence in $f(K)$ has a convergent subsequence whose limit is in $f(K)$, and therefore $f(K)$ is compact.     
\end{proof}

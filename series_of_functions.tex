\section{Series of Functions}
\begin{definition*}[pointwise convergence of a series] Let $f_n$ for each $n\in\mathbb{N}$ and $f$ be functions defined on a set $a\subseteq\mathbb{R}$.  The infinite series
\[
\sum_{n=1}^\infty f_n(x) = f_1(x)+f_2(x)+f_3(x)+\cdots
\]
converges pointwise on $A$ to $f(x)$ if the sequence $s_k(x)$ of partial sums,
\[
s_k(x) = f_1(x)+f_2(x)+\cdots+f_k(x)
\] 
converges pointwise to $f(x)$.
\par\vspace{0.3 cm}
The series converges uniformly to $f$ on $A$ if the sequence $s_k(x)$ converges uniformly to $f(x)$ on $A$. 
\end{definition*}
\par\vspace{0.4 cm}
Adapting the Cauchy criterion for sequences of functions to series:
\par\vspace{0.4 cm}
\begin{theorem}[Cauchy Criterion for Uniform Convergence of Series] A series
\[
\sum_{n=1}^\infty f_n
\]
converges uniformly on $A\subseteq\mathbb{R}$ if and only if, for every $\epsilon>0$, there exists an $N\in\mathbb{N}$ such that 
\[
|f_{m+1}(x)+f_{m+2}(x)+\cdots+f_n(x)| < \epsilon \quad\mbox{whenever}\quad m,n\geq N\quad\mbox{for all}\quad x\in A
\]
\end{theorem}
\par\vspace{0.4 cm}
\begin{theorem}[Weierstrass M-Test] For each $n\in\mathbb{N}$, let $f_n$ be a function defined on a set $A\subseteq\mathbb{R}$, and let $M_n>0$ be a real number satisfying
\[
|f_n(x)| \leq M_n
\]
for all $x\in A$.  If
\[
\sum_{n=1}^\infty M_n
\]
converges, then
\[
\sum_{n=1}^\infty f_n
\]
converges uniformly on $A$.
\end{theorem}
\par\vspace{0.4 cm}
\begin{proof}
Suppose 
\[
\sum_{n=1}^\infty M_n
\]
converges.
Let $\epsilon>0$ be given.  We need to show that there exists an $N\in\mathbb{N}$ such that
\[
|f_{m+1}(x)+f_{m+2}(x)+\cdots+f_n(x)| < \epsilon\quad\mbox{whenever}\quad m,n\geq N\quad\mbox{for all}\quad x\in A 
\]
By hypothesis, $\sum M_n$ converges, so by the Cauchy Criterion there is an $N\in\mathbb{N}$ such that
\[
M_{m+1}+M_{m+2}+\cdots+M_n < \epsilon\quad\mbox{whenever}\quad m,n \geq N
\] 
Then by the triangle inequality,
\[
|f_{m+1}(x)+f_{m+2}(x)+\cdots+f_n(x)| \leq |f_{m+1}(x)+f_{m+2}(x)+\cdots+f_{n-1}(x)|+|f_n(x)|
\]
\[
\leq |f_{m+1}(x)+f_{m+2}(x)+\cdots+f_{n-1}(x)|+M_n  
\]
Applying the triangle inequality again gives
\[
|f_{m+1}(x)+f_{m+2}(x)+\cdots+f_n(x)| \leq |f_{m+1}(x)+f_{m+2}(x)+\cdots+f_{n-2}(x)|+M_{n-1}+M_n
\]
And eventually
\[
|f_{m+1}(x)+f_{m+2}(x)+\cdots+f_n(x)| \leq M_{m+1}+M_{m+2}+\cdots+M_n < \epsilon\quad\mbox{whenever}\quad m,n\geq N
\]
for all $x\in A$.
\end{proof}
\begin{definition}[power series] A function of the form
\[
f(x) = \sum_{n=0}^\infty a_nx^n = a_0+a_1x+a_2x^2+a_3x^3+\cdots
\]
is called a power series.
\end{definition}
\begin{theorem}
The power series
\[
f(x) = \sum_{n=0}^\infty a_nx^n = a_0+a_1x+a_2x^2+a_3x^3+\cdots
\]
converges to $a_0$ if $x=0$.
\end{theorem}
\begin{proof}
The result follows by substitution of $x=0$ into the power series:
\[
f(0) = \sum_{n=0}^\infty a_nx^n = a_0+0+0+0+\cdots
\]
\end{proof}
\begin{theorem}
If a power series
\[
\sum_{n=0}^\infty a_nx^n
\]
converges at some point $x_0\in\mathbb{R}$ then it converges uniformly on the closed interval $[-c,c]$ where
\[
c = |x_0|
\]
\end{theorem}
\begin{proof}
Set
\[
M_n = |a_nx_0^n|
\]
By hypothesis, the series converges absolutely, so we can write
\[
\sum_{n=0}^\infty |a_nx_0^n| = \sum_{n=0}^\infty M_n
\]
so $\sum M_n$ converges.  For any $x\in[-c,c]$,
\[
|a_nx^n|\leq|a_nx_0^n|= M_n
\]
so the Weierstrass M-test guarantees that $\sum_{x=0}^\infty a_nx^n$ converges uniformly on $[-c,c]$. 
\end{proof}
